\documentclass[12pt]{article}
\usepackage[hidelinks]{hyperref}
\usepackage[dvipsnames]{xcolor}
\pagestyle{empty}
\begin{document}
\vspace*{-3cm}
\begin{center}
{\bf Statement of Research Philosophy}\\
Matthew Brigida, Ph.D.
\end{center}
\vspace*{0.25cm}
My present research is focused on high-frequency trading and market microstructure, using financial engineering to fund biomedical research, energy markets, and energy derivative and project valuation.  To analyze high-frequency trading I wrote a \href{https://github.com/Matt-Brigida/CME-FIX-FAST-Translator}{\color{Blue}{series of scripts}} which will translate raw CME messages from FIX/FAST format, and build a full orderbook.  This allows me to translate CME market depth data (which is every message sent to and from the exchange) into a full orderbook time-stamped to the millisecond.  Using this data I have a \href{https://papers.ssrn.com/sol3/papers.cfm?abstract_id=2848527}{\color{Blue}{research paper}} which investigates the speed and efficiency of the market reaction to the weekly Natural Gas Storage Report.  I also have a paper which models how book liquidity reacts to trades at the millisocond level.

An exciting area where I have two early working papers is in structuring portfolios of compounds used in cancer research in order to attract more money into this research. The problem is phase 1 funding is very hard to attract due to the risk return profile.  Each compound costs \$200 million to develop, and has only a 5\% probability of success.  If it succeeds the owner earns \$2 billion in years $11$ through $20$.  This is too risky for venture capital investors, so we are trying to structure portfolios of these compounds in order to match typical venture capital investments.  You can see more on research on this problem here: \href{http://cancerx.mit.edu/}{\color{Blue}{http://cancerx.mit.edu/}}.

% These areas represent many nascent topics in the academic literature.  For example, while FTR options are actively traded in economically meaningful amounts, there is yet very little research on these securities in the academic literature (and next to none in the finance literature).  By being somewhat early into these fields, I hope to make an impact over a long career.  
Much of my previous research has been in energy finance (electricity, natural gas, and crude oil markets).  This was driven by my interesting work at NextEra Energy, where I valued spark spread options.  In 2014 I published an analysis of the cointegrating relationship between natural gas and crude oil prices in $Energy\ Economics$.  This paper, titled `The Switching Relationship between Crude Oil and Natural Gas Prices', won the USAEE/IAEE `Best Working Paper Award for 2012' (there were over 150 competing papers).  In the analysis I allow the cointegrating equation to endogenously switch regimes, which controls for a unit-root in the relative price process. Ultimately the analysis showed there is a strong relative pricing relationship between the two commodities.  A following paper determined the causes of the regime switching.  I also have a paper under revise and resumbit at the {\it Journal of Commodity Markets} which decomposes volatility into the proportion caused by a set of factors using a state-space framework.

In addition to working in energy finance, I have also focused my empirical work on time series analysis incorporating unobserved variables.  Specifically, I have built a collection of self-written code to estimate Markov regime-switching and state-space models.  Because I have written the code myself, I have the flexibility to tailor a particular model to the underlying economics of the process.  

%As a note, my empirical work does not require data from costly proprietary databases -- I use mainly data from regional ISOs (independent system operators) and the US Department of Energy.

%I also have a working paper on the state-dependent relationship between natural gas prices and the number of rigs in operation (under review at $Energy\ Policy$). Further working papers of mine value FTR options, concern hedging volatility in natural gas prices, and compare portfolio performance measure rankings. 

I have also published research on informed trading prior to mergers and acquisitions.  In 2013 I published in $Quantitative\ Finance$ (along with Jeff Madura and Ariel Viale) a mathematical model of informed trading.  In 2012 I published papers on informed trading in the $Journal\ of\ Economics\ and$ \\ $ Business$ and the $Journal\ of\ Economic\ Practice$ (with Jeff Madura). 
\\
\\
Best regards,\\
\\
Matthew Brigida, Ph.D.\\
Associate Professor of Finance\\
Clarion University of Pennsylvania
\end{document}
%%% Local Variables:
%%% mode: latex
%%% TeX-master: t
%%% End:
