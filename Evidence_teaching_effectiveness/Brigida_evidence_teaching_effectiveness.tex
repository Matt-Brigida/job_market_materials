\documentclass[12pt]{article}
\title{Evidence of Teaching Effectiveness}
\author{Matthew Brigida, Ph.D.}
\date{}
\usepackage{float}
\usepackage{hyperref}
\begin{document}
\maketitle

While at Clarion I have received excellent reviews of my teaching from students, peers, and the College of Busniess Dean.  Evidence from each of these sources follows.

\section{Student Evaluations}

A summary of student evaluations of my teaching is below\footnote{I also have average scores for each class in an Excel spreadsheet, and can scan and email the originals}.  I have averaged greater than 4 on every aspect of my teaching \footnote{the only measure with an average below 4 is regarding student effort in the course.} Particularly note that I scored very highly for treating students with respect (average of 4.83).  I also scored very highly with respect to: grading fairly (4.78); being well prepared (4.63); and allowing time for questions (4.76).

\begin{center}
\begin{table}[H]
\caption{Summary of student evaluations.  The scores below are averages weighted by the number of students in each course. The average is over 28 courses, and 615 student evaluations.}
\begin{tabular}{p{4.5in}c} \hline \hline
Question \scriptsize $\left( \begin{array}{ccc}  {\rm Excellent}(5) & \cdots & {\rm Poor}(1) \\ {\rm Strongly Agree}(5) & \cdots & {\rm Strongly Disagree}(1) \end{array} \right)$ & Average \\ \hline \hline
Overall, I found the course content to be. & 4.25 \\ \hline
Overall, I found the quality of instruction to be. & 4.36 \\ \hline
Compared with other courses at Clarion University, how much effort did you have to put into this course? & 3.77 \\ \hline
The instructor increased my interest in the course content. & 4.18 \\ \hline
Assignments for this course increased my understanding of the subject.  & 4.37 \\ \hline
The instructor encouraged me to think in depth about the subject. & 4.44 \\ \hline
I learned a lot in this course. & 4.27 \\ \hline
The instructor clearly established course objectives. & 4.52 \\ \hline
The instructor was well prepared. & 4.63 \\ \hline
The course materials helped me achieve the course objectives. & 4.27 \\ \hline
The instructor has good communication skills. & 4.31 \\ \hline
The instructor used examples, illustrations, and applications that helped in understanding course material. & 4.59 \\ \hline
The instructor treats students with respect. & 4.83 \\ \hline
The instructor allows adequate opportunity for student questions in class. & 4.76 \\ \hline
The instructor's evaluation procedures were clearly established.  & 4.58 \\ \hline
Examination questions were clearly worded. & 4.47 \\ \hline
Examinations cover material or skills stressed in the course. & 4.66 \\ \hline
Exams and assignments were graded fairly. & 4.78 \\ \hline
\end{tabular}
\end{table}
\end{center}

\subsection{Student Projects}

In addition to standard teaching duties, I have created projects which allow students to apply financial and software skills.  For example, I have supervised work on a relational database of FTR and electricity prices, in order to allow students to practice the SQL language.  These projects enable students to put valuable software skills on their resume---which is of great help in their job searches.  

In addition to advising undergraduate finance students, I have also served as an advisor to the MBA program.  For the 2012--2013 academic year I was the sole advisor to all on-campus MBA students.  For the 2013--2014 year I will be the designated faculty advisor from the Finance department.

\section{Peer Evaluations}
The peer evaluations of my teaching have been very positive.  They have noted that I am well-prepared, have a thorough knowledge of the subject matter, am enthusiastic, and that I have been successful in eliciting student participation.  For example Jeff Eicher, Professor of Finance, notes, ``An extensive level of preparation was evidenced in the instructor's lecture and he combined the use of multiple teaching techniques and aides in his class.''\footnote{Peer evaluation: 9/12/2013} Soga Ewedemi, Professor of Finance, said, ``The instructor's command of the subject matter was great, handling the material with total comfort. He used the chalkboard extensively working through examples of problems.''. \footnote{Peer evaluation: 2/21/2011}

Particularly noted in my peer evaluations, is my use of online brokerage accounts which I have arranged for my students.  The accounts allow students to immediately apply what they have learned, and they report that this helps them retain the knowledge.

Lastly, in my annual review written by the Dean of the College of Business, Dean Pesek has said that my student evaluation scores are evidence that ``Dr. Brigida has carried out his teaching duties very effectively''\footnote{2010--2011 performance review}.  

I have retained all of my peer evaluations, and would be happy to scan and email them to you if you wish.

\subsection{Program Development}
Given my success in course development, I have assisted in the creation of three new programs at Clarion. The first is a Finance concentration within the MBA program.  I developed the curriculum for the program, and it has been approved through all levels of the state system of higher education.  The second program is a Master's degree in Data Analytics. I was the Coordinator of the program, where I set the curriculum and created 6 of the courses.  The program has been approved and is now accepting students.  The last program is an undergraduate major in Mathematical Finance, where I advised the Mathematics department on the curriculum.  This program has also been approved. 
\end{document}
